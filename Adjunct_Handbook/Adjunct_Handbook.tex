
\documentclass{book}

\usepackage{
  fancyhdr,    % custom chapter headings
  fullpage,    % reasonable margins
  hyperref,    % hotlinks
  makeidx,     % index
  ebgaramond        % font choice
}

% Don't say "Chapter 3", just "3"
\renewcommand{\chaptername}{}

\makeindex

\begin{document}

\title{Adjunct Handbook}
\author{Department of Computer Science \\ California State University, Fullerton}
\date{2015 Edition, Revision $\alpha$}
\maketitle

\newpage
\tableofcontents

\chapter{Introduction}

\section{The department}

\section{Syllabus}
checklist

expectation of how many hours a student should commit to work
"enrollment regulation" from the catalog under University regulations
http://catalog.fullerton.edu/content.php?catoid=2&navoid=112
ENROLLMENT REGULATIONS
Units Of Credit
Each semester unit represents three hours of university work per week for one semester. Courses are of three types:
Lecture: one hour in class plus two hours of study.
Activity: two hours of class plus one hour of study.
Laboratory: three hours of laboratory activity in class plus one hour of study outside class.
Some courses may combine two or more of these types. All required courses carry unit credit.

Auditing
under Registration Information
http://catalog.fullerton.edu/content.php?catoid=2&navoid=154
"Auditors
A properly qualified student may enroll in classes as an auditor. The student must meet the regular University admission requirements and must pay the same tuition and fees as other students. See the description of Audit in the “University Regulations” section of this catalog under “Administrative Symbols.”"
\section{Office hours}
http://www.fullerton.edu/senate/documents/UPS230.020_Office%20Hours_effec%205-11-12.pdf

\section{Assessment}
\subsection{Program assessment}
\subsection{Instructor assessment}

\subsection{Resources}
https://csuf-ecs.onthehub.com

\section{Student work}
Generally: one semester retention

UPS 320.005 (5/23/1989)
http://www.fullerton.edu/senate/documents/PDF/300/UPS320-005.pdf

Exams
The student has no right to the return of written work resulting from a test or examination. However, the student does have a right to examine and discuss tests and examinations with the faculty member involved, and at the option of the faculty member, these materials may be returned to the student. Examinations not returned to students shall be retained on file for one semester after the last day of the semester in which the course was taken except when they become part of an academic appeal in which case they shall be retained until the appeal has been concluded.

Research & Creative Activity (Homework)
It is the student's responsibility to request the return of products of research and creative activity. Where such products have been submitted for evaluation in fulfillment or partial fulfillment for a course, program, degree or other certification and where the student does not request the return of such products within one semester after the student's final course grades are assigned or after the date of the awarding of a degree or other certification, the faculty has no further obligation to retain such materials.

\chapter{Credits and Revision History}

Copyright 2015, Department of Computer Science, California State University, Fullerton.

\chapter{Index}
\printindex

\end{document}
